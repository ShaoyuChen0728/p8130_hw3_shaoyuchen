% Options for packages loaded elsewhere
\PassOptionsToPackage{unicode}{hyperref}
\PassOptionsToPackage{hyphens}{url}
%
\documentclass[
]{article}
\usepackage{amsmath,amssymb}
\usepackage{iftex}
\ifPDFTeX
  \usepackage[T1]{fontenc}
  \usepackage[utf8]{inputenc}
  \usepackage{textcomp} % provide euro and other symbols
\else % if luatex or xetex
  \usepackage{unicode-math} % this also loads fontspec
  \defaultfontfeatures{Scale=MatchLowercase}
  \defaultfontfeatures[\rmfamily]{Ligatures=TeX,Scale=1}
\fi
\usepackage{lmodern}
\ifPDFTeX\else
  % xetex/luatex font selection
\fi
% Use upquote if available, for straight quotes in verbatim environments
\IfFileExists{upquote.sty}{\usepackage{upquote}}{}
\IfFileExists{microtype.sty}{% use microtype if available
  \usepackage[]{microtype}
  \UseMicrotypeSet[protrusion]{basicmath} % disable protrusion for tt fonts
}{}
\makeatletter
\@ifundefined{KOMAClassName}{% if non-KOMA class
  \IfFileExists{parskip.sty}{%
    \usepackage{parskip}
  }{% else
    \setlength{\parindent}{0pt}
    \setlength{\parskip}{6pt plus 2pt minus 1pt}}
}{% if KOMA class
  \KOMAoptions{parskip=half}}
\makeatother
\usepackage{xcolor}
\usepackage[margin=1in]{geometry}
\usepackage{color}
\usepackage{fancyvrb}
\newcommand{\VerbBar}{|}
\newcommand{\VERB}{\Verb[commandchars=\\\{\}]}
\DefineVerbatimEnvironment{Highlighting}{Verbatim}{commandchars=\\\{\}}
% Add ',fontsize=\small' for more characters per line
\usepackage{framed}
\definecolor{shadecolor}{RGB}{248,248,248}
\newenvironment{Shaded}{\begin{snugshade}}{\end{snugshade}}
\newcommand{\AlertTok}[1]{\textcolor[rgb]{0.94,0.16,0.16}{#1}}
\newcommand{\AnnotationTok}[1]{\textcolor[rgb]{0.56,0.35,0.01}{\textbf{\textit{#1}}}}
\newcommand{\AttributeTok}[1]{\textcolor[rgb]{0.13,0.29,0.53}{#1}}
\newcommand{\BaseNTok}[1]{\textcolor[rgb]{0.00,0.00,0.81}{#1}}
\newcommand{\BuiltInTok}[1]{#1}
\newcommand{\CharTok}[1]{\textcolor[rgb]{0.31,0.60,0.02}{#1}}
\newcommand{\CommentTok}[1]{\textcolor[rgb]{0.56,0.35,0.01}{\textit{#1}}}
\newcommand{\CommentVarTok}[1]{\textcolor[rgb]{0.56,0.35,0.01}{\textbf{\textit{#1}}}}
\newcommand{\ConstantTok}[1]{\textcolor[rgb]{0.56,0.35,0.01}{#1}}
\newcommand{\ControlFlowTok}[1]{\textcolor[rgb]{0.13,0.29,0.53}{\textbf{#1}}}
\newcommand{\DataTypeTok}[1]{\textcolor[rgb]{0.13,0.29,0.53}{#1}}
\newcommand{\DecValTok}[1]{\textcolor[rgb]{0.00,0.00,0.81}{#1}}
\newcommand{\DocumentationTok}[1]{\textcolor[rgb]{0.56,0.35,0.01}{\textbf{\textit{#1}}}}
\newcommand{\ErrorTok}[1]{\textcolor[rgb]{0.64,0.00,0.00}{\textbf{#1}}}
\newcommand{\ExtensionTok}[1]{#1}
\newcommand{\FloatTok}[1]{\textcolor[rgb]{0.00,0.00,0.81}{#1}}
\newcommand{\FunctionTok}[1]{\textcolor[rgb]{0.13,0.29,0.53}{\textbf{#1}}}
\newcommand{\ImportTok}[1]{#1}
\newcommand{\InformationTok}[1]{\textcolor[rgb]{0.56,0.35,0.01}{\textbf{\textit{#1}}}}
\newcommand{\KeywordTok}[1]{\textcolor[rgb]{0.13,0.29,0.53}{\textbf{#1}}}
\newcommand{\NormalTok}[1]{#1}
\newcommand{\OperatorTok}[1]{\textcolor[rgb]{0.81,0.36,0.00}{\textbf{#1}}}
\newcommand{\OtherTok}[1]{\textcolor[rgb]{0.56,0.35,0.01}{#1}}
\newcommand{\PreprocessorTok}[1]{\textcolor[rgb]{0.56,0.35,0.01}{\textit{#1}}}
\newcommand{\RegionMarkerTok}[1]{#1}
\newcommand{\SpecialCharTok}[1]{\textcolor[rgb]{0.81,0.36,0.00}{\textbf{#1}}}
\newcommand{\SpecialStringTok}[1]{\textcolor[rgb]{0.31,0.60,0.02}{#1}}
\newcommand{\StringTok}[1]{\textcolor[rgb]{0.31,0.60,0.02}{#1}}
\newcommand{\VariableTok}[1]{\textcolor[rgb]{0.00,0.00,0.00}{#1}}
\newcommand{\VerbatimStringTok}[1]{\textcolor[rgb]{0.31,0.60,0.02}{#1}}
\newcommand{\WarningTok}[1]{\textcolor[rgb]{0.56,0.35,0.01}{\textbf{\textit{#1}}}}
\usepackage{graphicx}
\makeatletter
\def\maxwidth{\ifdim\Gin@nat@width>\linewidth\linewidth\else\Gin@nat@width\fi}
\def\maxheight{\ifdim\Gin@nat@height>\textheight\textheight\else\Gin@nat@height\fi}
\makeatother
% Scale images if necessary, so that they will not overflow the page
% margins by default, and it is still possible to overwrite the defaults
% using explicit options in \includegraphics[width, height, ...]{}
\setkeys{Gin}{width=\maxwidth,height=\maxheight,keepaspectratio}
% Set default figure placement to htbp
\makeatletter
\def\fps@figure{htbp}
\makeatother
\setlength{\emergencystretch}{3em} % prevent overfull lines
\providecommand{\tightlist}{%
  \setlength{\itemsep}{0pt}\setlength{\parskip}{0pt}}
\setcounter{secnumdepth}{-\maxdimen} % remove section numbering
\ifLuaTeX
  \usepackage{selnolig}  % disable illegal ligatures
\fi
\IfFileExists{bookmark.sty}{\usepackage{bookmark}}{\usepackage{hyperref}}
\IfFileExists{xurl.sty}{\usepackage{xurl}}{} % add URL line breaks if available
\urlstyle{same}
\hypersetup{
  pdftitle={sc5351\_hw3},
  pdfauthor={Shaoyu Chen},
  hidelinks,
  pdfcreator={LaTeX via pandoc}}

\title{sc5351\_hw3}
\author{Shaoyu Chen}
\date{2023-10-29}

\begin{document}
\maketitle

\begin{Shaded}
\begin{Highlighting}[]
\FunctionTok{library}\NormalTok{(tidyverse)}
\end{Highlighting}
\end{Shaded}

\begin{verbatim}
## -- Attaching core tidyverse packages ------------------------ tidyverse 2.0.0 --
## v dplyr     1.1.3     v readr     2.1.4
## v forcats   1.0.0     v stringr   1.5.0
## v ggplot2   3.4.3     v tibble    3.2.1
## v lubridate 1.9.2     v tidyr     1.3.0
## v purrr     1.0.2     
## -- Conflicts ------------------------------------------ tidyverse_conflicts() --
## x dplyr::filter() masks stats::filter()
## x dplyr::lag()    masks stats::lag()
## i Use the conflicted package (<http://conflicted.r-lib.org/>) to force all conflicts to become errors
\end{verbatim}

\begin{Shaded}
\begin{Highlighting}[]
\FunctionTok{library}\NormalTok{(dplyr)}
\FunctionTok{library}\NormalTok{(MASS)}
\end{Highlighting}
\end{Shaded}

\begin{verbatim}
## 
## Attaching package: 'MASS'
## 
## The following object is masked from 'package:dplyr':
## 
##     select
\end{verbatim}

\begin{Shaded}
\begin{Highlighting}[]
\NormalTok{df }\OtherTok{=}\NormalTok{ birthwt}
\end{Highlighting}
\end{Shaded}

\#\#Problem 1

Some medical professionals claim that the average weight of American
women is 171 pounds. The column lwt holds the mother's weight (in
pounds) at last menstrual period, i.e.~her pre-pregnancy weight. Use
this column for the following questions.

Problem 1a

Construct a 95\% confidence interval of true mean weight of American
women.

n = 189

\(\bar{X}\) (sample mean) = 129.8148148

\(s\) (sample standard deviation) = 30.5793804

\begin{Shaded}
\begin{Highlighting}[]
\NormalTok{n }\OtherTok{=} \FunctionTok{length}\NormalTok{(}\FunctionTok{pull}\NormalTok{(df, lwt))}
\NormalTok{x }\OtherTok{=} \FunctionTok{mean}\NormalTok{(}\FunctionTok{pull}\NormalTok{(df, lwt))}
\NormalTok{s }\OtherTok{=} \FunctionTok{sd}\NormalTok{(}\FunctionTok{pull}\NormalTok{(df, lwt))}
\end{Highlighting}
\end{Shaded}

\begin{Shaded}
\begin{Highlighting}[]
\NormalTok{lower\_lwt }\OtherTok{=}\NormalTok{ x }\SpecialCharTok{{-}} \FunctionTok{qt}\NormalTok{(}\FloatTok{0.975}\NormalTok{, }\AttributeTok{df =}\NormalTok{ n}\DecValTok{{-}1}\NormalTok{)}\SpecialCharTok{*}\NormalTok{(s}\SpecialCharTok{/}\FunctionTok{sqrt}\NormalTok{(n))}
\NormalTok{upper\_lwt }\OtherTok{=}\NormalTok{ x }\SpecialCharTok{+} \FunctionTok{qt}\NormalTok{(}\FloatTok{0.975}\NormalTok{, }\AttributeTok{df =}\NormalTok{ n}\DecValTok{{-}1}\NormalTok{)}\SpecialCharTok{*}\NormalTok{(s}\SpecialCharTok{/}\FunctionTok{sqrt}\NormalTok{(n))}
\end{Highlighting}
\end{Shaded}

95\% confidence interval is (125.43, 134.2).

Problem 1b

Interpret the confidence interval.

We are 95\% confident that the true population mean of women's weight
lies between (125.43, 134.2).

Problem 1c

Comment on the validity of the statement above (``Some medical
professionals claim that the average weight of American women is 171
pounds''). In other words, what can we say about this statement given
our confidence interval from part a?

This statement is not valid, because the average weight of American
women is 171 pounds is not included in the 95\% confidence interval.

\#\#Problem 2

In this data set, we have a variable (smoke) indicating the smoking
status of the mothers during pregnancy. Some doctors believe that
smoking status is related to weight. Using the columns smoke and lwt,
test this claim. (Note: a value of 1 indicates the mother is in the
``smoking'' group.)

Problem 2a

\begin{enumerate}
\def\labelenumi{\alph{enumi})}
\tightlist
\item
  Test for the equality of variances between the two groups. (Use a 5\%
  significance level.)
\end{enumerate}

Test for Equality of Variances

Sample 1: \(s_1, ~ n_1, ~ \bar{x}_1\)

Sample 2: \(s_2, ~ n_2, ~ \bar{x}_2\)

Testing the hypotheses:
\[  H_0 : \sigma_1^2 = \sigma_2^2 \quad vs \quad H_1 : \sigma_1^2 \ne \sigma_2^2\]

With significance level a pre-specified, compute the test statistic:
\[ F = {\frac{s_1^2}{s_2^2}} \sim F_{n_1-1, n_2-1},under H_0\]

\begin{Shaded}
\begin{Highlighting}[]
\NormalTok{var\_smoke }\OtherTok{=} 
\NormalTok{  birthwt }\SpecialCharTok{|\textgreater{}} 
  \FunctionTok{filter}\NormalTok{(smoke }\SpecialCharTok{==} \DecValTok{1}\NormalTok{) }\SpecialCharTok{|\textgreater{}} 
  \FunctionTok{pull}\NormalTok{(lwt) }\SpecialCharTok{|\textgreater{}}
  \FunctionTok{var}\NormalTok{()}

\NormalTok{var\_nonsmoke }\OtherTok{=} 
\NormalTok{  birthwt }\SpecialCharTok{|\textgreater{}} 
  \FunctionTok{filter}\NormalTok{(smoke }\SpecialCharTok{==} \DecValTok{0}\NormalTok{) }\SpecialCharTok{|\textgreater{}} 
  \FunctionTok{pull}\NormalTok{(lwt) }\SpecialCharTok{|\textgreater{}}
  \FunctionTok{var}\NormalTok{()}

\CommentTok{\# calculate F statistic}
\NormalTok{F\_test }\OtherTok{=}\NormalTok{ var\_smoke}\SpecialCharTok{/}\NormalTok{var\_nonsmoke}

\NormalTok{n\_smoke }\OtherTok{=} 
\NormalTok{  birthwt }\SpecialCharTok{|\textgreater{}} 
  \FunctionTok{filter}\NormalTok{(smoke }\SpecialCharTok{==} \DecValTok{1}\NormalTok{) }\SpecialCharTok{|\textgreater{}} 
  \FunctionTok{nrow}\NormalTok{()}
  
\NormalTok{n\_nonsmoke }\OtherTok{=} 
\NormalTok{  birthwt }\SpecialCharTok{|\textgreater{}} 
  \FunctionTok{filter}\NormalTok{(smoke }\SpecialCharTok{==} \DecValTok{0}\NormalTok{) }\SpecialCharTok{|\textgreater{}} 
  \FunctionTok{nrow}\NormalTok{()}

\CommentTok{\# calculate critical value  }
\NormalTok{F\_crit }\OtherTok{=} \FunctionTok{qf}\NormalTok{(}\FloatTok{0.975}\NormalTok{, }\AttributeTok{df1 =} \DecValTok{73}\NormalTok{, }\AttributeTok{df2 =} \DecValTok{114}\NormalTok{)}

\CommentTok{\# decision}
\FunctionTok{ifelse}\NormalTok{(F\_test }\SpecialCharTok{\textgreater{}}\NormalTok{ F\_crit, }\StringTok{"reject"}\NormalTok{, }\StringTok{"fail to reject"}\NormalTok{)}
\end{Highlighting}
\end{Shaded}

\begin{verbatim}
## [1] "fail to reject"
\end{verbatim}

So, it fail to reject

Problem 2b

Given your answer from part a, what kind of hypothesis test will you
perform?

I will perform Two-Sample Independent t-test.

Problem 2c

Conduct your chosen hypothesis test from part b at the 10\% significance
level. What is your decision regarding the null? Interpret this result
in the context of the problem.

Assume the hypotheses:
\[H_0 : \mu_1 = \mu_2 \quad vs \quad H_1 : \mu_1 \ne \mu_2\]

With significance level α pre-specified, compute the test statistic:
\[t = {\frac{\overline{X_1}-\overline{X_2}}{s\sqrt{\frac{1}{n_1}+\frac{1}{n_2}}}} \sim t_{n_1+n_2-2}\]

Where s is called the pooled sample standard deviation

\[s^2 = \frac{(n_1-1)s_1^2+(n_2-1)s_2^2}{n_1+n_2-2}\]

\begin{Shaded}
\begin{Highlighting}[]
\CommentTok{\# calculate mean X1,X2}
\NormalTok{mean\_nonsmoke }\OtherTok{=} 
\NormalTok{  birthwt }\SpecialCharTok{|\textgreater{}} 
  \FunctionTok{filter}\NormalTok{(smoke }\SpecialCharTok{==} \DecValTok{0}\NormalTok{) }\SpecialCharTok{|\textgreater{}} 
  \FunctionTok{pull}\NormalTok{(lwt) }\SpecialCharTok{|\textgreater{}}
  \FunctionTok{mean}\NormalTok{()}

\NormalTok{mean\_smoke }\OtherTok{=} 
\NormalTok{  birthwt }\SpecialCharTok{|\textgreater{}} 
  \FunctionTok{filter}\NormalTok{(smoke }\SpecialCharTok{==} \DecValTok{1}\NormalTok{) }\SpecialCharTok{|\textgreater{}} 
  \FunctionTok{pull}\NormalTok{(lwt) }\SpecialCharTok{|\textgreater{}}
  \FunctionTok{mean}\NormalTok{()}

\CommentTok{\# calculate s1,s2}
\NormalTok{sd\_nonsmoke }\OtherTok{=} 
\NormalTok{  birthwt }\SpecialCharTok{|\textgreater{}} 
  \FunctionTok{filter}\NormalTok{(smoke }\SpecialCharTok{==} \DecValTok{0}\NormalTok{) }\SpecialCharTok{|\textgreater{}} 
  \FunctionTok{pull}\NormalTok{(lwt) }\SpecialCharTok{|\textgreater{}}
  \FunctionTok{sd}\NormalTok{()}

\NormalTok{sd\_smoke }\OtherTok{=} 
\NormalTok{  birthwt }\SpecialCharTok{|\textgreater{}} 
  \FunctionTok{filter}\NormalTok{(smoke }\SpecialCharTok{==} \DecValTok{1}\NormalTok{) }\SpecialCharTok{|\textgreater{}} 
  \FunctionTok{pull}\NormalTok{(lwt) }\SpecialCharTok{|\textgreater{}}
  \FunctionTok{sd}\NormalTok{()}

\CommentTok{\# calculate s}
\NormalTok{s\_pool }\OtherTok{=} \FunctionTok{sqrt}\NormalTok{(((n\_nonsmoke}\DecValTok{{-}1}\NormalTok{)}\SpecialCharTok{*}\NormalTok{sd\_nonsmoke}\SpecialCharTok{\^{}}\DecValTok{2}\SpecialCharTok{+}\NormalTok{(n\_smoke}\DecValTok{{-}1}\NormalTok{)}\SpecialCharTok{*}\NormalTok{sd\_smoke}\SpecialCharTok{\^{}}\DecValTok{2}\NormalTok{)}\SpecialCharTok{/}\NormalTok{(n\_nonsmoke }\SpecialCharTok{+}\NormalTok{ n\_smoke }\SpecialCharTok{{-}}\DecValTok{2}\NormalTok{))}

\CommentTok{\# calculate T statistic}
\NormalTok{T\_test }\OtherTok{=}\NormalTok{ (mean\_nonsmoke }\SpecialCharTok{{-}}\NormalTok{ mean\_smoke)}\SpecialCharTok{/}\NormalTok{(s\_pool}\SpecialCharTok{*}\FunctionTok{sqrt}\NormalTok{((}\DecValTok{1}\SpecialCharTok{/}\NormalTok{n\_nonsmoke)}\SpecialCharTok{+}\NormalTok{(}\DecValTok{1}\SpecialCharTok{/}\NormalTok{n\_smoke)))}
\NormalTok{T\_test}
\end{Highlighting}
\end{Shaded}

\begin{verbatim}
## [1] 0.6047303
\end{verbatim}

\begin{Shaded}
\begin{Highlighting}[]
\CommentTok{\# calculate critical value  }
\NormalTok{T\_crit }\OtherTok{=} \FunctionTok{qt}\NormalTok{(}\FloatTok{0.95}\NormalTok{, n\_smoke }\SpecialCharTok{+}\NormalTok{ n\_nonsmoke }\SpecialCharTok{{-}} \DecValTok{2}\NormalTok{)}
\NormalTok{T\_crit}
\end{Highlighting}
\end{Shaded}

\begin{verbatim}
## [1] 1.653043
\end{verbatim}

\begin{Shaded}
\begin{Highlighting}[]
\CommentTok{\# decision}
\FunctionTok{ifelse}\NormalTok{(T\_test }\SpecialCharTok{\textgreater{}}\NormalTok{ T\_crit, }\StringTok{"reject"}\NormalTok{, }\StringTok{"fail to reject"}\NormalTok{)}
\end{Highlighting}
\end{Shaded}

\begin{verbatim}
## [1] "fail to reject"
\end{verbatim}

Based on this result, it fail to reject the null hypothesis, since
T\_test = 0.6047303 \textless{} T\_crit = 1.653043.

\#\#Problem 3

According to the CDC, approximately 20\% of pregnant American women
suffer from hypertension. Do our data support this claim? (Use column ht
- a value of 1 means the mother is suffering from hypertension.)

Problem 3a

Conduct a 99\% confidence interval and interpret the results. What can
we conclude about the CDC's claim from this interval?

It is Interval Estimation: One-Sample Proportion

a 100 (1 − a) \% confidence interval for one population proportion is
given by:
\[(\hat{p}-z_{1-a/2}\sqrt{\frac{\hat{p}(1-\hat{p})}{n}},\hat{p}+z_{1-a/2}\sqrt{\frac{\hat{p}(1-\hat{p})}{n}})\]

\begin{Shaded}
\begin{Highlighting}[]
\CommentTok{\# calculate number of mother is suffering from hypertension}
\NormalTok{n\_ht }\OtherTok{=} 
\NormalTok{  birthwt}\SpecialCharTok{|\textgreater{}}
  \FunctionTok{nrow}\NormalTok{()}

\NormalTok{ht\_0 }\OtherTok{=} 
\NormalTok{  birthwt}\SpecialCharTok{|\textgreater{}}
  \FunctionTok{filter}\NormalTok{(ht }\SpecialCharTok{==} \DecValTok{0}\NormalTok{) }\SpecialCharTok{|\textgreater{}}
  \FunctionTok{nrow}\NormalTok{()}

\NormalTok{ht\_1 }\OtherTok{=}
\NormalTok{  birthwt}\SpecialCharTok{|\textgreater{}}
  \FunctionTok{filter}\NormalTok{(ht }\SpecialCharTok{==} \DecValTok{1}\NormalTok{) }\SpecialCharTok{|\textgreater{}}
  \FunctionTok{nrow}\NormalTok{()}

\CommentTok{\# calculate the value of proportion}
\NormalTok{ht\_prop }\OtherTok{=}\NormalTok{ ht\_1}\SpecialCharTok{/}\NormalTok{(ht\_1 }\SpecialCharTok{+}\NormalTok{ ht\_0)}

\CommentTok{\# calculate the N(0,1) with a = 0.01}
\NormalTok{z\_ht }\OtherTok{=} \FunctionTok{qnorm}\NormalTok{(}\FloatTok{0.995}\NormalTok{, }\DecValTok{0}\NormalTok{, }\DecValTok{1}\NormalTok{)}

\CommentTok{\# calculate the lower and upper of CI}
\NormalTok{lower\_ht }\OtherTok{=}\NormalTok{ ht\_prop }\SpecialCharTok{{-}}\NormalTok{ z\_ht }\SpecialCharTok{*} \FunctionTok{sqrt}\NormalTok{(ht\_prop}\SpecialCharTok{*}\NormalTok{(}\DecValTok{1} \SpecialCharTok{{-}}\NormalTok{ ht\_prop)}\SpecialCharTok{/}\NormalTok{n\_ht)}
\NormalTok{upper\_ht }\OtherTok{=}\NormalTok{ ht\_prop }\SpecialCharTok{+}\NormalTok{ z\_ht }\SpecialCharTok{*} \FunctionTok{sqrt}\NormalTok{(ht\_prop}\SpecialCharTok{*}\NormalTok{(}\DecValTok{1} \SpecialCharTok{{-}}\NormalTok{ ht\_prop)}\SpecialCharTok{/}\NormalTok{n\_ht)}
\end{Highlighting}
\end{Shaded}

99\% confidence interval is (0.0178041, 0.10918).

We are 99\% confident that the population proportion of pregnant
American women suffer from hypertension lies between (0.0178041,
0.10918). Based on this result, the claim that CDC has approximately
20\% of pregnant American women suffer from hypertension is not correct,
because 0.2 does not include in 99\% confidence interval.

Problem 3b

Conduct a one-sided hypothesis test at the a = 0.1 level. In this test,
we want to see if the true proportion is indeed less than the claimed
20\%. What can we conclude about the CDC's claim?

I will use One-Sample Test for Binomial Proportion,tests for
One-Population Proportion, Normal Theory Methods

Assume the hypothesis: \[H_0 : p = p_0 \quad vs \quad H_1 : p < p_0\]

With significance level α pre-specified, compute the test statistic:
\[ z = {\frac{\hat{p} - p_0}{\sqrt{p_0(1 - p_0)/n}}}\sim N(0,1)\]

\begin{Shaded}
\begin{Highlighting}[]
\CommentTok{\# calculate critical value  }
\NormalTok{Z\_crit }\OtherTok{=} \FunctionTok{qnorm}\NormalTok{(}\FloatTok{0.95}\NormalTok{, }\DecValTok{0}\NormalTok{, }\DecValTok{1}\NormalTok{)}
\NormalTok{Z\_crit}
\end{Highlighting}
\end{Shaded}

\begin{verbatim}
## [1] 1.644854
\end{verbatim}

\begin{Shaded}
\begin{Highlighting}[]
\CommentTok{\# calculate Z statistic}
\NormalTok{Z\_test }\OtherTok{=}\NormalTok{ (ht\_prop}\FloatTok{{-}0.2}\NormalTok{)}\SpecialCharTok{/}\FunctionTok{sqrt}\NormalTok{(}\FloatTok{0.2}\SpecialCharTok{*}\NormalTok{(}\DecValTok{1}\FloatTok{{-}0.2}\NormalTok{)}\SpecialCharTok{/}\NormalTok{n\_ht)}
\NormalTok{Z\_test}
\end{Highlighting}
\end{Shaded}

\begin{verbatim}
## [1] -4.691685
\end{verbatim}

\begin{Shaded}
\begin{Highlighting}[]
\CommentTok{\# decision}
\FunctionTok{ifelse}\NormalTok{(}\FunctionTok{abs}\NormalTok{(Z\_test) }\SpecialCharTok{\textgreater{}}\NormalTok{ Z\_crit, }\StringTok{"reject"}\NormalTok{, }\StringTok{"fail to reject"}\NormalTok{)}
\end{Highlighting}
\end{Shaded}

\begin{verbatim}
## [1] "reject"
\end{verbatim}

Based on this result, \textbar Z\_test\textbar{} = 4.691685
\textgreater{} Z\_crit = 1.644854, so it reject the null hypothesis,
which means the true proportion is indeed less than the claimed 20\%.
for CDC's claim, approximately 20\% of pregnant American women suffer
from hypertension is correct.

\#\#Problem 4

Is there a difference between uterine irritability in the group of
pregnant women who smoke vs the group of pregnant women that don't
smoke? (Use columns ui and smoke.) Conduct a hypothesis test at the a =
0.01 level. What can we conclude about the proportions of women with
uterine irritability between the smoking groups?

First I will use Tests for Two-Population Proportions, Normal Theory
Methods:

Assume the hypothesis:
\[H_0 : p_1 = p_2 \quad vs \quad H_1 : p_1 \ne p_2\]

The test statistic is given by:
\[z = {\frac{\hat{p_1} - \hat{p_2}}{\sqrt{\hat{p}\hat{q}(\frac{1}{n_1}+\frac{1}{n_2})}}}\sim N(0,1)\]
and where \[\hat{p} = \frac{n_1\hat{p_1}+n_2\hat{p_2}}{n_1+n_2}\]

\begin{Shaded}
\begin{Highlighting}[]
\CommentTok{\# calculate p1, p2 and p}
\NormalTok{smoke\_ui }\OtherTok{=} 
\NormalTok{  birthwt}\SpecialCharTok{|\textgreater{}}
  \FunctionTok{filter}\NormalTok{(smoke }\SpecialCharTok{==} \DecValTok{1}\NormalTok{) }\SpecialCharTok{|\textgreater{}} 
  \FunctionTok{group\_by}\NormalTok{(ui) }\SpecialCharTok{|\textgreater{}}
  \FunctionTok{filter}\NormalTok{(ui }\SpecialCharTok{==} \DecValTok{1}\NormalTok{) }\SpecialCharTok{|\textgreater{}}
  \FunctionTok{nrow}\NormalTok{()}

\NormalTok{nonsmoke\_ui }\OtherTok{=} 
\NormalTok{  birthwt}\SpecialCharTok{|\textgreater{}}
  \FunctionTok{filter}\NormalTok{(smoke }\SpecialCharTok{==} \DecValTok{0}\NormalTok{) }\SpecialCharTok{|\textgreater{}} 
  \FunctionTok{group\_by}\NormalTok{(ui) }\SpecialCharTok{|\textgreater{}}
  \FunctionTok{filter}\NormalTok{(ui }\SpecialCharTok{==} \DecValTok{1}\NormalTok{) }\SpecialCharTok{|\textgreater{}}
  \FunctionTok{nrow}\NormalTok{()}

\CommentTok{\# number of nonsmoke = n1, number of smoke = n2}
\NormalTok{n1 }\OtherTok{=} 
\NormalTok{  birthwt }\SpecialCharTok{|\textgreater{}}
  \FunctionTok{filter}\NormalTok{(smoke }\SpecialCharTok{==} \DecValTok{0}\NormalTok{) }\SpecialCharTok{|\textgreater{}}
  \FunctionTok{nrow}\NormalTok{()}

\NormalTok{n2 }\OtherTok{=} 
\NormalTok{  birthwt }\SpecialCharTok{|\textgreater{}}
  \FunctionTok{filter}\NormalTok{(smoke }\SpecialCharTok{==} \DecValTok{1}\NormalTok{) }\SpecialCharTok{|\textgreater{}}
  \FunctionTok{nrow}\NormalTok{()}

\CommentTok{\# calculate p1 and p2 }
\NormalTok{p1 }\OtherTok{=}\NormalTok{ nonsmoke\_ui }\SpecialCharTok{/}\NormalTok{ n1}
\NormalTok{p2 }\OtherTok{=}\NormalTok{ smoke\_ui }\SpecialCharTok{/}\NormalTok{ n2}

\CommentTok{\# calculate p}
\NormalTok{p }\OtherTok{=}\NormalTok{ (n1}\SpecialCharTok{*}\NormalTok{p1}\SpecialCharTok{+}\NormalTok{n2}\SpecialCharTok{*}\NormalTok{p2) }\SpecialCharTok{/}\NormalTok{ (n1 }\SpecialCharTok{+}\NormalTok{ n2)}

\CommentTok{\# calculate Z statistic}
\NormalTok{Z\_test2 }\OtherTok{=}\NormalTok{ (p1 }\SpecialCharTok{{-}}\NormalTok{ p2) }\SpecialCharTok{/} \FunctionTok{sqrt}\NormalTok{(p}\SpecialCharTok{*}\NormalTok{(}\DecValTok{1}\SpecialCharTok{{-}}\NormalTok{p)}\SpecialCharTok{*}\NormalTok{((}\DecValTok{1}\SpecialCharTok{/}\NormalTok{n1)}\SpecialCharTok{+}\NormalTok{(}\DecValTok{1}\SpecialCharTok{/}\NormalTok{n2)))}
\NormalTok{Z\_test2}
\end{Highlighting}
\end{Shaded}

\begin{verbatim}
## [1] -0.8545449
\end{verbatim}

\begin{Shaded}
\begin{Highlighting}[]
\CommentTok{\# calculate critical value }
\NormalTok{Z\_crit2 }\OtherTok{=} \FunctionTok{qnorm}\NormalTok{(}\FloatTok{0.995}\NormalTok{, }\DecValTok{0}\NormalTok{, }\DecValTok{1}\NormalTok{)}
\NormalTok{Z\_crit2}
\end{Highlighting}
\end{Shaded}

\begin{verbatim}
## [1] 2.575829
\end{verbatim}

\begin{Shaded}
\begin{Highlighting}[]
\CommentTok{\# decision}
\FunctionTok{ifelse}\NormalTok{(}\FunctionTok{abs}\NormalTok{(Z\_test2) }\SpecialCharTok{\textgreater{}}\NormalTok{ Z\_crit2, }\StringTok{"reject"}\NormalTok{, }\StringTok{"fail to reject"}\NormalTok{)}
\end{Highlighting}
\end{Shaded}

\begin{verbatim}
## [1] "fail to reject"
\end{verbatim}

According to this result, At 0.01 significance level, we can't reject
the null hypothesis, because \textbar Z\_test2\textbar{} = 0.8545449
\textless{} Z\_crit2 = 2.575829. we can say that there is not a
significant difference between proportions of women with uterine
irritability between the smoking groups.

\#\#Problem 5

Is race related to birth weight? (Use columns race and bwt.)

\begin{Shaded}
\begin{Highlighting}[]
\NormalTok{race\_df }\OtherTok{=}
\NormalTok{  birthwt }\SpecialCharTok{|\textgreater{}} 
  \FunctionTok{group\_by}\NormalTok{(race) }\SpecialCharTok{|\textgreater{}} 
  \FunctionTok{count}\NormalTok{(race)}
\NormalTok{race\_df}
\end{Highlighting}
\end{Shaded}

\begin{verbatim}
## # A tibble: 3 x 2
## # Groups:   race [3]
##    race     n
##   <int> <int>
## 1     1    96
## 2     2    26
## 3     3    67
\end{verbatim}

Problem 5a:

What test would be most appropriate to answer this question?

ANOVA test.

Problem 5b

What assumptions are we making by using this test? Are all assumptions
met?

\begin{enumerate}
\def\labelenumi{\arabic{enumi}.}
\tightlist
\item
  There are k populations of interest (k \textgreater{} 2),
\item
  The samples are drawn independently from the underlying populations;
\item
  Homoscedasticity: the variances of the k populations are equal;
\item
  Normality: the distributions of the error terms are normal (could be
  relaxed when the sample size is large) They are all met!!! k = 3, the
  variances of the k populations are equal, the distributions of the
  error terms are normal, and the samples are drawn independently from
  the underlying populations.
\end{enumerate}

Problem 5c

Conduct the test at the 5\% significance level and interpret your
results. Be sure to write the hypotheses you are testing.

Assume the hypothesis:

\[H_0 : \mu_1 = \mu_2 = \mu_3 \quad vs \quad H_1 : at least two means are not equal\]

Compute the test statistic:

\[F = {\frac{BetweenSS/(k-1)}{WithinSS/(n-k)}}\sim F_{k-1,n-k}\]

distribution under \(H_0\)

Rejection rules: Reject \(H_0\): if F \textgreater{} \(F_{k-1,n-k,1-a}\)

Fail to reject \(H_0\): if F \(\leqslant F_{k-1,n-k,1-a}\)

Problem 5d

Perform multiple comparisons - which races are significantly different?
Interpret your results

\begin{Shaded}
\begin{Highlighting}[]
\FunctionTok{pairwise.t.test}\NormalTok{(birthwt}\SpecialCharTok{$}\NormalTok{bwt,birthwt}\SpecialCharTok{$}\NormalTok{race,}\AttributeTok{p.adj=}\StringTok{"bonferroni"}\NormalTok{)}
\end{Highlighting}
\end{Shaded}

\begin{verbatim}
## 
##  Pairwise comparisons using t tests with pooled SD 
## 
## data:  birthwt$bwt and birthwt$race 
## 
##   1     2    
## 2 0.049 -    
## 3 0.029 1.000
## 
## P value adjustment method: bonferroni
\end{verbatim}

Based on this result, race 1 is different.

\end{document}
